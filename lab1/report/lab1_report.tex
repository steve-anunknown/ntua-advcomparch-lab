\documentclass{article}
\usepackage[english,greek, main=greek]{babel}
\usepackage[utf8]{inputenc}
\usepackage{fullpage}

\usepackage{amsmath}
\usepackage{chngcntr}
\counterwithin{equation}{section}

\usepackage{graphicx}
\usepackage{subcaption}
\usepackage{placeins}

\usepackage{multirow}
\usepackage{xcolor}
\usepackage{listings}

\definecolor{mGreen}{rgb}{0,0.6,0}
\definecolor{mGray}{rgb}{0.5,0.5,0.5}
\definecolor{mPurple}{rgb}{0.58,0,0.82}
\definecolor{light-gray}{gray}{0.95}
\definecolor{backgroundColour}{rgb}{0.95,0.95,0.92}

\lstdefinestyle{CStyle}{
    backgroundcolor=\color{backgroundColour},   
    commentstyle=\color{mGreen},
    keywordstyle=\color{magenta},
    numberstyle=\tiny\color{mGray},
    stringstyle=\color{mPurple},
    basicstyle=\footnotesize,
    breakatwhitespace=false,         
    breaklines=true,                 
    captionpos=b,                    
    keepspaces=true,                 
    numbers=left,                    
    numbersep=5pt,                  
    showspaces=false,                
    showstringspaces=false,
    showtabs=false,                  
    tabsize=2,
    language=C
}

\lstset{language=bash, 
        basicstyle=\small\ttfamily, 
        keywordstyle=\color{blue}\bfseries, 
        commentstyle=\color{gray}, 
        stringstyle=\color{green!50!black},
        backgroundcolor=\color{light-gray},
        showstringspaces=false,
        breaklines=true,
        linewidth=\textwidth}

\lstset{language=Python,
        basicstyle=\small\ttfamily,
        keywordstyle=\bfseries\color{blue},
        commentstyle=\color{gray},
        stringstyle=\color{green!50!black},
        showstringspaces=false,
        breaklines=true}

\newcommand{\eng}[1]{\foreignlanguage{english}{#1}}
\newcommand{\Alpha}{\mathrm{A}}

\useshorthands{;}
\defineshorthand{;}{?}

\title{Προηγμένα Θέματα Αρχιτεκτονικής Υπολογιστών\\
\large Άσκηση 1η}
\author{Αναστάσιος Στέφανος Αναγνώστου \\
\large 03119051}

\begin{document}

\maketitle

\clearpage
\tableofcontents
\clearpage

Σημειώνεται, ότι πριν την έναρξη των πειραμάτων, πρέπει να πλοηγηθεί κανείς στο αρχείο \texttt{\eng{cache.h}} και να εντοπίσει τους κατασκευαστές των αντικειμένων \texttt{\eng{cache}}. Εκεί, πρέπει να αλλαχθούν οι προκαθορισμένες τιμές ποινής αστοχίας ώστε να ταυτίζονται με τις ζητούμενες από την εκφώνηση της άσκησης.

\section{\eng{L1 Cache}}

Στο πρώτο πείραμα δοκίμαζονται διάφορες τιμές των παραμέτρων του πρώτου επιπέδου της κρυφής μνήμης, διατηρώντας σταθερές τις παραμέτρους του δευτέρου επιπέδου και του \eng{TLB}. Συγκεκριμένα, δοκιμάζονται οι τιμές:

\selectlanguage{english}
\begin{table}[h]
    \centering
    \begin{tabular}{|p{3cm}|p{3cm}|p{3cm}|}
        \hline
        L1 size (KB) & L1 associativity & L1 block size (KB)\\
        \hline
        32 & 4 & 64\\
        \hline    
        32 & 8 & 32, 64, 128\\
        \hline
        64 & 4 & 64\\
        \hline
        64, 128 & 8 & 32, 64, 128\\
        \hline
    \end{tabular}
\end{table}
\selectlanguage{greek}

και διατηρούνται σταθερές οι τιμές:

\selectlanguage{english}
\begin{table}[h]
    \centering
    \begin{tabular}{|p{3cm}|p{3cm}|p{3cm}|}
        \hline
        L2 size (KB) & L2 associativity & L2 block size (KB)\\
        \hline
        \hline
        1024 & 8 & 128\\
        \hline
        TLB size (entries) & TLB associativity & TLB page size (B)\\
        \hline    
        \hline
        64 & 4 & 4096\\
        \hline
    \end{tabular}
\end{table}
\selectlanguage{greek}

Παρακάτω, φαίνονται οι εντολές ανά κύκλο που επιτεύχθησαν στα διάφορα μετροπρογράμματα με τις παραπάνω διατάξειες. Σημειώνεται ότι οι διατάξεις δοκιμάστηκαν και διατάσσονται στα γραφήματα όπως παρατίθενται παραπάνω.


\graphicspath{{../parsec-3.0/parsec_workspace/outputs/exp1/}}
\clearpage
\subsection{\eng{Blackscholes}}

Επιθεωρώντας το γράφημα του εκτελέσιμου \eng{blackscholes} \ref{fig:exp1-blackscholes}, φαίνεται ότι η αύξηση της συσχετιστικότητας $4 \rightarrow 8$ συνεισέφερε στην αύξηση του \eng{IPC}. Στην συνέχεια, η αύξηση του μεγέθους του μπλοκ συνεισφέρει επίσης στην αύξηση του \eng{IPC}, αλλά όχι στον ίδιο βαθμό. Όταν επαναφέρεται η συσχετιστικότητα  $8 \rightarrow 4$ και δοκιμάζεται διπλάσιο μέγεθος \eng{L1}, η απόδοση πέφτει και αυξάνεται εκ νέου με την αύξηση της συσχετιστικότητας.  Σε αυτό το σημείο είναι σαφές ότι, τουλάχιστον για αυτό το εκτελέσιμο, η συσχετιστικότητα 8 αποδίδει καλύτερα. Διπλασιάζοντας εκ νέου το μέγεθος \eng{L1}, σταθερής της συσχετιστικότητας, δεν επιτυγχάνεται το ίδιο καλή απόδοση όπως προηγουμένως αλλά παρατηρείται η ίδια αύξηση του \eng{IPC} μεγαλώνοντας το μέγεθος του μπλοκ. Σημειώνεται, βέβαια, ότι η απόδοση είναι εν πάση περιπτώσει μικρή, στο 20\%.

\begin{figure}[h]
    \centering
    \includegraphics[width=0.6\textwidth]{./blackscholes/L1.png} 
    \caption{\eng{IPC} στο εκτελέσιμο \eng{blackscholes} συναρτήσει διαφόρων παραμέτρων της ιεραρχίας μνήμης}
    \label{fig:exp1-blackscholes}
\end{figure}
\FloatBarrier

\clearpage
\subsection{\eng{Bodytrack}}

Στο εκτελέσιμο \eng{bodytrack} παρατηρείται σαφώς καλύτερη απόδοση, τάξεως 60\%. Εν προκειμένω, φαίνεται, ότι η αύξηση του μεγέθους \eng{L1} προκαλεί αύξηση στην απόδοση, η οποία ενισχύεται και από την αύξηση του μεγέθους μπλοκ, ίσως με εξαίρεση το μέγεθος \eng{cache} 128 ΚΒ, όπου σημειώνεται μεν η μέγιστη απόδοση για συσχετιστικότητα 8 αλλά σημειώνεται μείωση με την αύξηση του μεγέθους μπλοκ.

\begin{figure}[h]
    \centering
    \includegraphics[width=0.6\textwidth]{./bodytrack/L1.png} 
    \caption{\eng{IPC} στο εκτελέσιμο \eng{bodytrack} συναρτήσει διαφόρων παραμέτρων της ιεραρχίας μνήμης}
    \label{fig:exp1-bodytrack}
\end{figure}
\FloatBarrier

\clearpage
\subsection{\eng{Canneal}}

Στο γράφημα \ref{fig:exp1-canneal} διακρίνεται αρχικά εξαιρετικά χαμηλό \eng{IPC}, τάξεως 1\%.  Στην συνέχεια, παρατηρείται πτώση με την αύξηση της συσχετιστικότητας. Τα ακραία μεγέθη της μνήμης \eng{L1} (32 ΚΒ, 128 ΚΒ) σημειώνουν καλύτερη απόδοση, ευνοούμενη περαιτέρω από την αύξηση του μεγέθους μπλοκ. Τέλος, παρατηρείται μία ιδιομορφία στα 64 ΚΒ χωρητικότητας μνήμης και 128Β μπλοκ, όπου σημειώνεται ξαφνική αύξηση και η βέλτιστη απόδοση εκ των δοκιμασμένων παραμέτρων, παρότι προηγουμένως σημειώνονταν κακές αποδόσεις.

\begin{figure}[h]
    \centering
    \includegraphics[width=0.6\textwidth]{./canneal/L1.png} 
    \caption{\eng{IPC} στο εκτελέσιμο \eng{canneal} συναρτήσει διαφόρων παραμέτρων της ιεραρχίας μνήμης}
    \label{fig:exp1-canneal}
\end{figure}
\FloatBarrier

\clearpage
\subsection{\eng{Fluidanimate}}

Στο γράφημα \ref{fig:exp1-fluidanimate} διακρίνεται αρχικά σχετικά υψηλό \eng{IPC}, τάξεως 60\%. Παρατηρείται επαναλαμβανόμενη συμπεριφορά στην απόδοση. Συγκεκριμένα, η περιοδική αυτή συμπεριφορά φαίνεται να προκαλεί άμεση απεικόνιση της επίδρασης των μεταβολών του μεγέθους του μπλοκ. Οι τιμές του μεγέθους του μπλοκ μεταβάλλονται ως $64 \rightarrow 32 \rightarrow 64 \rightarrow 128 \rightarrow 64 \rightarrow 32 ...$ και οι τιμές της απόδοσης ακολουθούν την μονοτονία του μεγέθους του μπλοκ, δηλαδή αύξηση οδηγεί σε αύξηση και μείωση σε μείωση. Η αύξηση ενισχύεται περεταίρω από την αύξηση του μεγέθους της \eng{L1}, για αυτό και παρατηρείται μέγιστη απόδοση στις μέγιστες τιμές 128 ΚΒ, 128 Β.

\begin{figure}[h]
    \centering
    \includegraphics[width=0.6\textwidth]{./fluidanimate/L1.png} 
    \caption{\eng{IPC} στο εκτελέσιμο \eng{fluidanimate} συναρτήσει διαφόρων παραμέτρων της ιεραρχίας μνήμης}
    \label{fig:exp1-fluidanimate}
\end{figure}
\FloatBarrier

\clearpage
\subsection{\eng{Freqmine}}

Στο γράφημα \ref{fig:exp1-freqmine} διακρίνεται αρχικά σχετικά υψηλό \eng{IPC}, τάξεως 60\%. Η απόδοση είναι αρκετά σταθερή παρόλες τις μεταβολές των παραμέτρων. Αξιοσημειώτες είαι δύο ιδιομορφίες, οι οποίες σημειώνονται αμφότερες όταν αυξάνεται το μέγεθος του μπλοκ σε 128Β. Κατά τ'άλλα, επικρατεί σταθερότητα.

\begin{figure}[h]
    \centering
    \includegraphics[width=0.6\textwidth]{./freqmine/L1.png} 
    \caption{\eng{IPC} στο εκτελέσιμο \eng{freqmine} συναρτήσει διαφόρων παραμέτρων της ιεραρχίας μνήμης}
    \label{fig:exp1-freqmine}
\end{figure}
\FloatBarrier

\clearpage
\subsection{\eng{Rtview}}

Στο γράφημα \ref{fig:exp1-rtview} παρατηρείται παρόμοια συμπεριφορά με το γράφημα \ref{fig:exp1-fluidanimate}, τόσο στην περιοδικότητα όσο και στην τάξη μεγέθους, η οποία είναι ελαφρώς μεγαλύτερο στο 70\%. Παρατηρείται, πάλι, μέγιστη απόδοση με την αύξηση του μεγέθους της \eng{L1} και του μεγέθους του μπλοκ.

\begin{figure}[h]
    \centering
    \includegraphics[width=0.6\textwidth]{./rtview/L1.png} 
    \caption{\eng{IPC} στο εκτελέσιμο \eng{rtview} συναρτήσει διαφόρων παραμέτρων της ιεραρχίας μνήμης}
    \label{fig:exp1-rtview}
\end{figure}
\FloatBarrier

\clearpage
\subsection{\eng{streamcluster}}

Στο γράφημα \ref{fig:exp1-streamcluster} παρατηρείται παρόμοια συμπεριφορά με τα γραφήματα \ref{fig:exp1-fluidanimate} και \ref{fig:exp1-rtview}, μόνο στην περιοδικότητα, όμως, και όχι στην υάξη μεγέθους, η οποία είναι χαμηλή στο 20\%. Παρατηρείται, πάλι, μέγιστη απόδοση με την αύξηση του μεγέθους της \eng{L1} και του μεγέθους του μπλοκ.

\begin{figure}[h]
    \centering
    \includegraphics[width=0.6\textwidth]{./streamcluster/L1.png} 
    \caption{\eng{IPC} στο εκτελέσιμο \eng{streamcluster} συναρτήσει διαφόρων παραμέτρων της ιεραρχίας μνήμης}
    \label{fig:exp1-streamcluster}
\end{figure}
\FloatBarrier

\clearpage
\subsection{\eng{swaptions}}

Στο γράφημα \ref{fig:exp1-swaptions} σημειώνεται απόδοση τάξεως 70\%. Σαφής είναι η εξάρτηση από το μέγεθος της \eng{L1} και το μέγεθος μπλοκ, τα οποία προκαλούν αύξηση καθώς τα ίδια αυξάνονται. Η αύξηση της συσχετιστικότητας φαίνεται να προκαλεί μείωση όταν το μέγεθος της μνήμης είναι μικρό, φαινόμενο ακόμα ασθενέστερο για μεγαλύτερα μεγέθη μνήμης και μπλοκ. Παρατηρείται, πάλι, μέγιστη απόδοση με την αύξηση του μεγέθους της \eng{L1} και του μεγέθους του μπλοκ.

\begin{figure}[h]
    \centering
    \includegraphics[width=0.6\textwidth]{./swaptions/L1.png} 
    \caption{\eng{IPC} στο εκτελέσιμο \eng{swaptions} συναρτήσει διαφόρων παραμέτρων της ιεραρχίας μνήμης}
    \label{fig:exp1-swaptions}
\end{figure}
\FloatBarrier

Κάτι αξιοσημείωτο είναι, ότι παρ'όλες τις διάφορες διακυμάνσεις, δεν προκαλείται αλλαγή στην τάξη μεγέθους της απόδοσης, καθώς διατηρείται η πρώτη σχετικά σταθερή. Γενικώς, η φανερώτερη εξάρτηση φαίνεται να είναι η αυξητική τάση του \eng{IPC} ως προς την αύξηση του μεγέθους μπλοκ. Τέλος, παρατηρείται, ότι γενικώς η συμπεριφορά του \eng{IPC} είναι η αντίστροφη από του \eng{MPKI}.

\clearpage
\section{\eng{L2 Cache}}

Στο δεύτερο πείραμα, δοκιμάζονται διάφορες τιμές των παραμέτρων του δευτέρου επιπέδου της κρυφής μνήμης, διατηρώντας σταθερές τις παραμέτρους του πρώτου επιπέδου της κρυφής μνήμης και του \eng{TLB}. Συγκεκριμένα, δοκιμάζονται οι τιμές:

\selectlanguage{english}
\begin{table}[h]
    \centering
    \begin{tabular}{|p{3cm}|p{3cm}|p{3cm}|}
        \hline
        L2 size (KB) & L2 associativity & L2 block size (KB)\\
        \hline
        512 & 8 & 64, 128, 256\\
        \hline    
        1024 & 8, 16 & 64, 128, 256\\
        \hline
        2048 & 16 & 64, 128, 256\\
        \hline
    \end{tabular}
\end{table}
\selectlanguage{greek}

και διατηρούνται σταθερές οι τιμές:

\selectlanguage{english}
\begin{table}[h]
    \centering
    \begin{tabular}{|p{3cm}|p{3cm}|p{3cm}|}
        \hline
        L1 size (KB) & L1 associativity & L1 block size (KB)\\
        \hline
        \hline
        32 & 8 & 64\\
        \hline
        TLB size (entries) & TLB associativity & TLB page size (B)\\
        \hline    
        \hline
        64 & 4 & 4096\\
        \hline
    \end{tabular}
\end{table}
\selectlanguage{greek}

\graphicspath{{../parsec-3.0/parsec_workspace/outputs/exp2/}}

\subsection{\eng{Blackscholes}}

Επιθεωρώντας το γράφημα του εκτελέσιμου \eng{blackscholes} \ref{fig:exp2-blackscholes}, φαίνεται ότι η αύξηση της απόδοσης εξαρτάται σε μεγάλο βαθμό από το μέγεθος του μπλοκ. Συγκεκριμένα, παρότι η απόδοση με το πρώτο σύνολο παραμέτρων (μέγεθος μπλοκ 64 Β) ήταν τάξεως 10\%, αμέσως η αύξηση του μεγέθους μπλοκ οδήγησε σε διπλασιασμό της και η περεταίρω αύξησή του ενέτεινε ακόμα, όχι στον ίδιο βαθμό, την απόδοση. Η αλλαγή του μεγέθους της μνήμης (και η επαναφορά του μικρού μεγέθους μπλοκ) έριξε την απόδοση αλλά οι ακόλουθες αυξήσεις την επανέφεραν. Το ίδιο μοτίβο παρατηρείται και στην συνέχεια, με μέγιστη απόδοση να επιτυγχάνεται τελικά για τις μέγιστες τιμές χωρητικότητας και μεγέθους μπλοκ.

\begin{figure}[h]
    \centering
    \includegraphics[width=0.6\textwidth]{./blackscholes/L2.png} 
    \caption{\eng{IPC} στο εκτελέσιμο \eng{blackscholes} συναρτήσει διαφόρων παραμέτρων της ιεραρχίας μνήμης}
    \label{fig:exp2-blackscholes}
\end{figure}
\FloatBarrier

\subsection{\eng{Bodytrack}}

Στο εκτελέσιμο \eng{bodytrack} παρατηρείται σαφώς καλύτερη απόδοση, τάξεως 60\%. Η αύξηση του μεγέθους του μπλοκ πάλι φαίνεται να ευνοεί την απόδοση, αφού παρατηρούνται αυξήσεις της όταν αυτό αυξάνεται αλλά και απότομες πτώσεις όταν αυτό λαμβάνει την ελάχιστη τιμή του. Συνολικά, η απόδοση φαίνεται να ευνοείται και από της χωρητικότητας και της συσχετιστικότητας, αφού έχει γενικώς αυξητική τάση. Η μέγιστη απόδοση λαμβάνεται για μέγιστες τιμές συσχετιστικότητας, χωρητικότητας και μεγέθους μπλοκ, με εμφανή διαφορά. 

\begin{figure}[h]
    \centering
    \includegraphics[width=0.6\textwidth]{./bodytrack/L2.png} 
    \caption{\eng{IPC} στο εκτελέσιμο \eng{bodytrack} συναρτήσει διαφόρων παραμέτρων της ιεραρχίας μνήμης}
    \label{fig:exp2-bodytrack}
\end{figure}
\FloatBarrier

\clearpage
\subsection{\eng{Canneal}}

Στο γράφημα \ref{fig:exp2-canneal} διακρίνεται αρχικά εξαιρετικά χαμηλό \eng{IPC}, τάξεως μικρότερης του 1\%. Ακόμα μια φορά, παρατηρείται γενικώς αυξητική τάση. Η αύξηση του μεγέθους μπλοκ του δευτέρου επιπέδου ευνοεί, γεγονός φανερό από τις πτώσεις της απόδοσης στην επαναφορά του μεγέθους του μπλοκ στην ελάχιστη τιμή. Η γενικώς αυξητική τάση δείχνει ότι η απόδοση ευνοείται επίσης από την αύξηση του μεγέθους του δευτέρου επιπέδου μνήμης. Η αύξηση της συσχετιστικότητας ενδεχομένως να έχει θετική επίδραση, αλλά η επίδραση του μεγέθους της μνήμης είναι μεγαλύτερη. Αυτό φαίνεται από την απότομη αύξηση της απόδοσης στο τέλος, όπου η συσχετιστικότητα παραμένει σταθερή στο 16 ενώ το μέγεθος της μνήμης διπλασιάζεται. 

\begin{figure}[h]
    \centering
    \includegraphics[width=0.6\textwidth]{./canneal/L2.png} 
    \caption{\eng{IPC} στο εκτελέσιμο \eng{canneal} συναρτήσει διαφόρων παραμέτρων της ιεραρχίας μνήμης}
    \label{fig:exp2-canneal}
\end{figure}
\FloatBarrier

\clearpage
\subsection{\eng{Fluidanimate}}

Στο γράφημα \ref{fig:exp2-fluidanimate} διακρίνεται αρχικά σχετικά υψηλό \eng{IPC}, τάξεως 60\%. Παρατηρείται επαναλαμβανόμενη συμπεριφορά στην απόδοση. Συγκεκριμένα, η περιοδικότητα φαίνεται να οφείλεται, πάλι, στις τιμές του μεγέθους μπλοκ. Αυτήν την φορά, η επίδραση της χωρητικότητας της μνήμης δεν είναι τόσο ισχυρή, αφού η απόδοση λαμβάνει τις ίδιες τιμές παρ'όλες τις αυξήσεις της χωρητικότητας. Η αύξηση της συσχετιστικότητας φαίνεται να συνεισφέρει σε έναν μικρό βαθμό. Η μέγιστη τιμή απόδοσης λαμβάνεται για μέγιστες τιμές μεγέθους μπλοκ, συσχετιστικότητας και χωρητικότητας.

\begin{figure}[h]
    \centering
    \includegraphics[width=0.6\textwidth]{./fluidanimate/L2.png} 
    \caption{\eng{IPC} στο εκτελέσιμο \eng{fluidanimate} συναρτήσει διαφόρων παραμέτρων της ιεραρχίας μνήμης}
    \label{fig:exp2-fluidanimate}
\end{figure}
\FloatBarrier

\clearpage
\subsection{\eng{Freqmine}}

Στο γράφημα \ref{fig:exp2-freqmine} διακρίνεται αρχικά σχετικά υψηλό \eng{IPC}, τάξεως 50-60\%. Στην αρχή παρουσιάζεται μία ελαφρώς αυξητική τάση προκαλούμενη από την αύξηση του μεγέθους του μπλοκ δευτέρου επιπέδου. Παρατηρείται μία μικρή μείωση όταν αυξάνεται η χωρητικότητα και επαναφέρεται στο ελάχιστο μέγεθος το μπλοκ, αλλά όχι μεγαλύτερη από την προηγούμενη συνολική αύξηση, πράγμα που σημαίνει ότι και η αύξηση της χωρητικότητας συμβάλει στην αύξηση της απόδοσης. Ωστόσο, στις τιμές παραμέτρων 1024 ΚΒ, 8, 128 Β, παρατηρείται μία ιδιομορφία και μειώνεται πολύ απότομα η απόδοση, επαναφερόμενη στην αναμενόμενη τιμή αμέσως. Η μέγιστη απόδοση επιτυγχάνεται,για ακόμα μία φορά, στις μέγιστες τιμές χωρητικότητας, συσχέτισης και μεγέθους μπλοκ.

\begin{figure}[h]
    \centering
    \includegraphics[width=0.6\textwidth]{./freqmine/L2.png} 
    \caption{\eng{IPC} στο εκτελέσιμο \eng{freqmine} συναρτήσει διαφόρων παραμέτρων της ιεραρχίας μνήμης}
    \label{fig:exp2-freqmine}
\end{figure}
\FloatBarrier

\clearpage
\subsection{\eng{Rtview} και \eng{steamcluster}}

Τα εκτελέσιμα \eng{rtview} και \eng{streamcluster} παρουσιάζονται μαζί γιατί έχουν όμοια συμπεριφορά. Συγκεκριμένα, με εξαίρεση την τάξη μεγέθους του \eng{IPC}, αντιστοίχως 70\% και 20\%, η συμπεριφορά τους στις μεταβολές των παραμέτρων είναι σχεδόν πανομοιότυπη. Συγκεκριμένα, αμφότερα φαίνονται να εξαρτώνται σε μεγάλο βαθμό από το μέγεθος του μπλοκ. Αξιοσημείωτη είναι η συμπεριφορά του \eng{streamcluster}, το οποίο φαίνεται να είναι εντελώς ανεπηρεάστο από τις μεταβολές στην χωρητικότητα και την συσχετιστικότητα της μνήμης. Το δε \eng{rtview} φαίνεται να εξαρτάται σε κάποιον βαθμό από την χωρητικότητα και την συσχετιστικότητα, αλλά όχι τόσο όσο από το μέγεθος μπλοκ, αφού γρήγορα σταθεροποιείται το εύρος των τιμών του \eng{IPC}. Το μέγιστη \eng{IPC} επιτυγχάνεται για μέγιστο μέγεθος μπλοκ, δηλαδή 256 Β, και χωρητικότητα μνήμης διάφορη των 512 ΚΒ.

\begin{figure}[h]
    \centering
    \begin{subfigure}{0.6\textwidth}
        \includegraphics[width=\textwidth]{./rtview/L2.png} 
        \caption{\eng{IPC} στο εκτελέσιμο \eng{rtview} συναρτήσει διαφόρων παραμέτρων της ιεραρχίας μνήμης}
        \label{fig:exp2-rtview}
    \end{subfigure}
    \begin{subfigure}{0.6\textwidth}
        \includegraphics[width=\textwidth]{./streamcluster/L2.png} 
        \caption{\eng{IPC} στο εκτελέσιμο \eng{streamcluster} συναρτήσει διαφόρων παραμέτρων της ιεραρχίας μνήμης}
        \label{fig:exp2-streamcluster}
    \end{subfigure}
\end{figure}
\FloatBarrier

\clearpage
\subsection{\eng{Swaptions}}

Στο γράφημα \ref{fig:exp2-swaptions} σημειώνεται απόδοση τάξεως 70\%, με πάρα πολύ μικρές διακυμάνσεις. Οι διακυμάνσεις είναι τόσο μικρές ώστε να μην έχει και τόσο νόημα να μιλήσει κανείς για εξάρτηση από κάποια παράμετρο. Ακόμα και η ιδιομορφία, αν και οπτικά φανερή, προκαλεί στο \eng{IPC} μία μεταβολή τάξεως 0.1\%. Δεν φαίνεται να υπάρχει σαφής εξάρτηση. Η μεταβολή τόσο στο \eng{IPC} όσο και στο \eng{MPKI} είναι εξαιρετικά μικρή.

\begin{figure}[h]
    \centering
    \includegraphics[width=0.6\textwidth]{./swaptions/L2.png} 
    \caption{\eng{IPC} στο εκτελέσιμο \eng{swaptions} συναρτήσει διαφόρων παραμέτρων της ιεραρχίας μνήμης}
    \label{fig:exp2-swaptions}
\end{figure}
\FloatBarrier

Συμπερασματικά, παρατηρείται αυξητική τάση του \eng{IPC} τόσο ως προς το μέγεθος του μπλοκ, όσο και ως προς το μέγεθος της μνήμης. Αυτή δεν παρατηρείται στον ίδιο βαθμό σε όλα τα εκτελέσιμα, καθώς μερικά εμφανίζουν παλινδρομική συμπεριφορά (\eng{rtview, streamcluster}) χωρίς σαφή τάση, ή ανωμαλίες \eng{freqmine}.


\clearpage
\section{\eng{TLB}}

Στο τρίτο πείραμα, δοκιμάζονται διάφορες τιμές των παραμέτρων του \eng{TLB}, διατηρώντας σταθερές τις τιμές των παραμέτρων του πρώτου και δευτέρου επιπέδου της κρυφής μνήμης. Συγκεκριμένα, δοκιμάζονται οι τιμές:

\selectlanguage{english}
\begin{table}[h]
    \centering
    \begin{tabular}{|p{3cm}|p{3cm}|p{3cm}|}
        \hline
        TLB size (entries) & TLB associativity & TLB page size (B)\\
        \hline
        64 & 1, 2, 4, 8, 16, 32, 64 & 4096\\
        \hline
        128, 256 & 4 & 4096\\
        \hline
    \end{tabular}
\end{table}
\selectlanguage{greek}

και διατηρούνται σταθερές οι τιμές:

\selectlanguage{english}
\begin{table}[h]
    \centering
    \begin{tabular}{|p{3cm}|p{3cm}|p{3cm}|}
        \hline
        L1 size (KB) & L1 associativity & L1 block size (KB)\\
        \hline
        \hline
        32 & 8 & 64\\
        \hline
        L2 size (KB) & L2 associativity & L2 block size (KB)\\
        \hline    
        \hline
        1024 & 8 & 128\\
        \hline
    \end{tabular}
\end{table}
\selectlanguage{greek}

\graphicspath{{../parsec-3.0/parsec_workspace/outputs/exp3/}}

\subsection{\eng{Blackscholes}}

Επιθεωρώντας το γράφημα του εκτελέσιμου \eng{blackscholes} \ref{fig:exp3-blackscholes}, φαίνεται ότι το \eng{IPC} αυξάνεται με την αύξηση της συσχετιστικότητας μέχρι και 4. Μετά μένει υψηλό και σταθερό μέχρι και την τιμή 64. Στην συνέχεια, η αύξηση του πλήθους των καταχωρήσεων το ρίχνει κατά πολύ.

\begin{figure}[h]
    \centering
    \includegraphics[width=0.8\textwidth]{./blackscholes/tlb.png} 
    \caption{\eng{IPC} στο εκτελέσιμο \eng{blackscholes} συναρτήσει διαφόρων παραμέτρων της ιεραρχίας μνήμης}
    \label{fig:exp3-blackscholes}
\end{figure}
\FloatBarrier

\subsection{\eng{Bodytrack}}

Στο εκτελέσιμο \eng{bodytrack} παρατηρείται αυξητική τάση του \eng{IPC} με την συσχετιστικότητα, μέχρι και την τιμή 2. Μετά, παραμένει σταθερό σχεδόν μέχρι και την τιμή 64. Τέλος, η αύξηση του πλήθους των καταχωρήσεων προκαλεί την σημαντική του αύξηση. 

\begin{figure}[h]
    \centering
    \includegraphics[width=0.8\textwidth]{./bodytrack/tlb.png} 
    \caption{\eng{IPC} στο εκτελέσιμο \eng{bodytrack} συναρτήσει διαφόρων παραμέτρων της ιεραρχίας μνήμης}
    \label{fig:exp3-bodytrack}
\end{figure}
\FloatBarrier

\subsection{\eng{Canneal}, \eng{fluidanimate}, \eng{freqmine}, \eng{rtview}, \eng{streamcluster} και \eng{swaptions}}

Τα εκτελέσιμα παρουσιάζονται μαζί γιατί έχουν πανομοιότυπη συμπεριφορά.

\begin{center}
    \includegraphics[width=0.49\textwidth]{./canneal/tlb.png} 
    \includegraphics[width=0.49\textwidth]{./fluidanimate/tlb.png} 
    \includegraphics[width=0.49\textwidth]{./freqmine/tlb.png} 
    \includegraphics[width=0.49\textwidth]{./rtview/tlb.png} 
    \includegraphics[width=0.49\textwidth]{./streamcluster/tlb.png} 
    \includegraphics[width=0.49\textwidth]{./swaptions/tlb.png} 
\end{center}

\clearpage
 Γενικώς, το \eng{IPC} εξαρτάται έντονα από την συσχετιστικότητα, καθώς παρατηρείται έντονη αυξητική μεταβολή του ακόμα και σε μικρές μεταβολές της. Συγκεκριμένα, σχεδόν σε όλες τις περιπτώσεις, μέχρι και την συσχετιστικότηατ 4 έχει σημειωθεί ήδη πολύ μεγάλη αύξηση του \eng{IPC} και οι υπόλοιπες μεταβολές προκαλούν μικροδιακυμάνσεις. Ελαφρώς διαφορετική συμπεριφορά έχει το εκτελέσιμο \eng{canneal}, το οποίο παρουσιάζει αύξουσα συμπεριφορά μέχρι και την συσχετιστικότητα 32, πέφτει ελαφρώς στην 64 και μετά αυξάνεται εκ νέου λόγω της αύξησης των καταχωρήσεων. Τέλος, φαίνεται καθαρά η αντίστροφη σχέση του \eng{mpki} και του \eng{ipc}. 

\clearpage
\section{Προανάκληση}

Στο τέταρτο πείραμα, διατηρούνται σταθερές οι παράμετροι όλης της ιεραρχίας μνήμης και υλοποιείται προανάκληση \eng{NEXT-N-LINE}. Συγκεκριμένα, οι παράμετροι είναι οι ακόλουθες:

\selectlanguage{english}
\begin{table}[h]
    \centering
    \begin{tabular}{|p{3cm}|p{3cm}|p{3cm}|}
        \hline
        L1 size (KB) & L1 associativity & L1 block size (KB)\\
        \hline
        \hline
        32 & 8 & 64\\
        L2 size (KB) & L2 associativity & L2 block size (KB)\\
        \hline
        \hline
        1024 & 8 & 128\\
        \hline
        TLB size (entries) & TLB associativity & TLB page size (B)\\
        \hline    
        \hline
        64 & 4 & 4096\\
        \hline
    \end{tabular}
\end{table}
\selectlanguage{greek}
\FloatBarrier

και συμπληρώνεται ο ακόλουθος κώδικας στο αρχείο \texttt{\eng{cache.h}}.

\selectlanguage{english}
\begin{lstlisting}[style=Cstyle]
// PREFETCHING
ADDRINT prefetch_addr = addr;
for (UINT32 i=0; i < _l2_prefetch_lines; i++) {
    prefetch_addr += L2BlockSize();
    /* .......................... */
    /* Add here prefetching code. */
    /* .......................... */
    CACHE_TAG prefetch_tag;
    UINT32 prefetch_set_index;
    SplitAddress(prefetch_addr, L2LineShift(), L2SetIndexMask(), prefetch_tag, prefetch_set_index);
    SET& prefetch_set = _l2_sets[prefetch_set_index];
    bool prefetch_hit = prefetch_set.Find(prefetch_tag);
    if (!prefetch_hit) {
        CACHE_TAG l2_replaced = prefetch_set.Replace(prefetch_tag);
        // If L2 is inclusive and a TAG has been replaced we need to remove
        // all evicted blocks from L1.
        if ((L2_INCLUSIVE == 1) && !(l2_replaced == INVALID_TAG)) {
        ADDRINT replacedAddr = ADDRINT(l2_replaced) << FloorLog2(L2NumSets());
        replacedAddr = replacedAddr | prefetch_set_index;
        replacedAddr = replacedAddr << L2LineShift();
        for (UINT32 i=0; i < L2BlockSize(); i+=L1BlockSize()) {
            ADDRINT newAddr = replacedAddr | i;
            SplitAddress(newAddr, L1LineShift(), L1SetIndexMask(), l1Tag, l1SetIndex);
            l1Set = _l1_sets[l1SetIndex];
            l1Set.DeleteIfPresent(l1Tag);
            }
        }
    }
}

\end{lstlisting}
\selectlanguage{greek}

\subsection{Εκτελέσιμα}

Αρχικά, παρατηρείται ότι η τάξη μεγέθους του \eng{IPC}, σε όλα τα εκτελέσιμα, είναι περίπου 20\%. Γενικώς, η αύξηση του πλήθους προανακληνωμένων γραμμών προκαλεί αύξηση στο \eng{IPC}. Μόνο στα εκτελέσιμα \eng{blackscholes, bodytrack} και \eng{swaptions} παρατηρούνται λίγες ανωμαλίες, όπως οι απότομες πτώσεις και η άμεση ανάκαμψη στα δύο πρώτα εκτελέσιμα  και η ηπιότερη πτώση και άμεση ανάκαμψη στο τελευταιό. Φαίνεται, επίσης, η αναμενόμενη συσχέτιση μεταξύ της πτώσης του \eng{mpki} των δύο επιπέδων μνήμης και της αύξησης του \eng{ipc}. Αξιοσημειώτο είναι, ότι, παρά την αρχική αύξηση των αστοχιών του πρώτου επιπέδου κρυφής μνήμης, το \eng{IPC} αυξάνεται. Αυτό οφείλεται στην μείωση των αστοχιών του δευτέρου επιπέδου κρυφής μνήμης, αφού είναι ακριβότερες και, άρα, η μείωσή τους είναι πιο επωφελής.

\graphicspath{{../parsec-3.0/parsec_workspace/outputs/exp4/}}
\begin{center}
    \includegraphics[width=.49\textwidth]{./blackscholes/prefetch.png} 
    \includegraphics[width=.49\textwidth]{./bodytrack/prefetch.png} 
    \includegraphics[width=.49\textwidth]{./canneal/prefetch.png} 
    \includegraphics[width=.49\textwidth]{./fluidanimate/prefetch.png} 
    \includegraphics[width=.49\textwidth]{./freqmine/prefetch.png} 
    \includegraphics[width=.49\textwidth]{./rtview/prefetch.png} 
    \includegraphics[width=.49\textwidth]{./streamcluster/prefetch.png} 
    \includegraphics[width=.49\textwidth]{./swaptions/prefetch.png} 
\end{center}
\FloatBarrier

Τα συμπεράσματα συνοψίζονται και στο παρακάτω γράφημα:

\begin{figure}[h]
    \includegraphics[width=\textwidth]{./ipc-prefetching.png}
    \caption{Μέση τιμή \eng{IPC} ανά πλήθος προανακληνώμενων γραμμών ανά πρόγραμμα}    
    \label{fig:exp4-avgs}
\end{figure}

Στο γράφημα \ref{fig:exp4-avgs} φαίνεται η μέση τιμή του \eng{IPC} ανά πλήθος προανακληνώμενων γραμμών ανά πρόγραμμα. Αποτυπώνεται η παραπάνω διαπίστωση, ότι δηλαδή όσο περισσότερες γραμμές προανακαλούνται τόσο μεγαλύτερο το \eng{IPC}.

\section{Προσαρμογή Κύκλου Ρολογιού}

Προηγουμένος υποτέθηκε ότι οι αλλαγές στην αρχιτεκτονική της ιεραρχίας μνήμης δεν επιφέρουν αλλαγές στην διάρκεια του κύκλου του επεξεργαστή. Αυτό, όμως, δεν ισχυεί. Για αυτόν τον λόγο, για καθέναν διπλασιασμό χωρητικότητας θα υποτεθεί 15\% αύξηση της διάρκειας του κύκλου, ενώ για κάθε διπλασιασμό της συσχετιστικότητας 5\% αύξηση.

Έστω, $D, C, T$ η χρονική διάρκεια του προγράμματος, το πλήθος των κύκλων και η διάρκεια του κύκλου αντιστοίχως. Θα επιχειρηθεί να αποδειχθεί ότι η προσαρμογή της διάρκειας του κύκλου ισοδυναμεί με προσαρμογή του πλήθους των κύκλων σταθερού του κύκλου, αρκεί να αυξηθεί καταλλήλως η συνολική διάρκεια εκτέλεσης.

\begin{equation}
    \begin{gathered}
        D = C * T \\
        IPC = \frac{I}{C} = \frac{I * T}{D} = T * \frac{I}{D} \\
    \end{gathered}
\end{equation}

Επιχειρείται ο υπολογισμός του προσαρμοσμένου \eng{IPC}, για έναν διπλασιασμό χωρητικότητας.

\begin{equation}
    \begin{gathered}
        D_{adj} = C * adj(T) = C * ( T + \frac{15}{100} * T) \Leftrightarrow \\
        D_{adj} = C * T * (1 + \frac{15}{100}) = C * T * \frac{115}{100} \Rightarrow \\
        IPC_{adj} = T * \frac{I}{D_{adj}} = T * \frac{I}{ C * T * \frac{115}{100}} \Leftrightarrow \\
        IPC_{adj} = \frac{100}{115} * IPC
    \end{gathered}
\end{equation}

Άρα, αν $n$ είναι το πλήθος των διπλασιασμών χωρητικότητας και $m$ το πλήθος διπλασιασμών συσχετιστικότητας, το προσαρμοσμένο \eng{IPC} είναι:

\begin{equation}
    IPC_{adj} = \left(\frac{100}{105}\right)^{n}\left(\frac{100}{115}\right)^{m} IPC
\end{equation}

Επομένως, είναι δυνατόν να προσαρμοστούν οι προηγούμενες μετρήσεις απλώς κλιμακώνοντάς τες καταλλήλως.

\subsection{\eng{L1 Cache} και \eng{L2 Cache}}

Στα πειράματα σχετικά με τα δύο επίπεα της κρυφής μνήμης γίνονταν διαφοροποιήσεις κυρίως στο μέγεθος του μπλοκ και στην χωρητικότητα. Η συσχετιστικότητα είτε διπλασιαζόταν είτε έμενε σταθερή. Επομένως, δεδομένου μάλιστα ότι ο διπλασιασμός της χωρητικότητας προκαλεί μεγαλύτερη αύξηση του κύκλου, αναμένεται πτώση των καταγεγραμμένων \eng{IPC}. Ωστόσο, αναμένεται να είναι φανερή η θετική επίδραση της αύξησης του μεγέθους μπλοκ.

\graphicspath{{../parsec-3.0/parsec_workspace/outputs/exp1/}}
\begin{center}
    \includegraphics[width=.49\textwidth]{./blackscholes/ipc_adjusted.png} 
    \includegraphics[width=.49\textwidth]{./bodytrack/ipc_adjusted.png} 
    \includegraphics[width=.49\textwidth]{./canneal/ipc_adjusted.png} 
    \includegraphics[width=.49\textwidth]{./fluidanimate/ipc_adjusted.png} 
    \includegraphics[width=.49\textwidth]{./freqmine/ipc_adjusted.png} 
    \includegraphics[width=.49\textwidth]{./rtview/ipc_adjusted.png} 
    \includegraphics[width=.49\textwidth]{./streamcluster/ipc_adjusted.png} 
    \includegraphics[width=.49\textwidth]{./swaptions/ipc_adjusted.png} 
\end{center}
\FloatBarrier

Πράγματι, κατά τη διάρκεια του πρώτου πειράματος παρατηρείται πτώση στην επίδοση του συστήματος, πράγμα λογικό αφού αυξάνεται συνεχώς η χωρητικότητα του πρώτου επιπέδου μνήμης. Παρ'όλα αυτά, όταν η χωρητικότητα και η συσχετιστικότητα παραμένουν σταθερές, η αύξηση του μεγέθους μπλοκ συνεχίζει να προκαλεί αύξηση στην επίδοση. 

Στο δεύτερο πείραμα, είχε παρατηρηθεί μεγαλύτερη εξάρτηση από το μέγεθος μπλοκ απ'ό,τι στο πρώτο πείραμα. Επομένως, αναμένωνονται ποιοτικά όμοιες γραφικές παραστάσεις, με κλιμακωμένες τις τιμές \eng{IPC}. 

\graphicspath{{../parsec-3.0/parsec_workspace/outputs/exp2/}}
\begin{center}
    \includegraphics[width=.49\textwidth]{./blackscholes/ipc_adjusted.png} 
    \includegraphics[width=.49\textwidth]{./bodytrack/ipc_adjusted.png} 
    \includegraphics[width=.49\textwidth]{./canneal/ipc_adjusted.png} 
    \includegraphics[width=.49\textwidth]{./fluidanimate/ipc_adjusted.png} 
    \includegraphics[width=.49\textwidth]{./freqmine/ipc_adjusted.png} 
    \includegraphics[width=.49\textwidth]{./rtview/ipc_adjusted.png} 
    \includegraphics[width=.49\textwidth]{./streamcluster/ipc_adjusted.png} 
    \includegraphics[width=.49\textwidth]{./swaptions/ipc_adjusted.png} 
\end{center}
\FloatBarrier

Πράγματι, στα εκτελέσιμα στα οποία το \eng{IPC} δεν έδειχνε μεγάλες μεταβολές, η κλιμάκωση είναι έντονη, όπως για παράδειγμα στο εκτελέσιμο \eng{swaptions} ή στο \eng{canneal}. Αντιθέτως, σε όσα έδειχναν έντονες διακυμάνσεις συναρτήσει του μεγέθους μπλοκ, όπως το \eng{blackscholes} και το \eng{streamcluster}, η κλιμάκωση δεν επιφέρει τόσο μεγάλη αλλαγή.  

\subsection{\eng{TLB}}

Στο πείραμα σχετικό με το \eng{TLB}, είχε φανεί ότι η συσχετιστικότητα επιδρούσε άμεσα και σε μεγάλο βαθμό στην επίδοση του συστήματος. Μάλιστα, στα περισσότερα εκτελέσιμα, η συσχετιστικότητα είχε ήδη μεγιστοποιήσει την επίδρασή της στον πρώτο ή στον δεύτερο διπλασιασμό. Μικρότερη αλλά υπαρκτή επιρροή είχε επίσης το πλήθος των καταχωρήσεων στο \eng{TLB}. Επομένως, αναμένεται η κατά πολύ ελάττωση της αυξητικής επίδρασης της συσχετιστικότητας στην επίδοση του συστήματος.

\graphicspath{{../parsec-3.0/parsec_workspace/outputs/exp3/}}
\begin{center}
    \includegraphics[width=.49\textwidth]{./blackscholes/ipc_adjusted.png} 
    \includegraphics[width=.49\textwidth]{./bodytrack/ipc_adjusted.png} 
    \includegraphics[width=.49\textwidth]{./canneal/ipc_adjusted.png} 
    \includegraphics[width=.49\textwidth]{./fluidanimate/ipc_adjusted.png} 
    \includegraphics[width=.49\textwidth]{./freqmine/ipc_adjusted.png} 
    \includegraphics[width=.49\textwidth]{./rtview/ipc_adjusted.png} 
    \includegraphics[width=.49\textwidth]{./streamcluster/ipc_adjusted.png} 
    \includegraphics[width=.49\textwidth]{./swaptions/ipc_adjusted.png} 
\end{center}
\FloatBarrier

Πράγματι, παρατηρείται ότι τα σημεία των γραφικών παραστάσεων στα οποία είχε πλέον διπλασιαστεί πολλαπλές φορές η συσχετιστικότητα παρουσιάζουν χαμηλές τιμές \eng{IPC}. Πολλά εκτελέσιμα απέδιδαν καλύτερα για χαμηλές τιμές συσχετιστικότητας, όπως 2 ή 4, και δεν βελτιώνονταν για υψηλές τιμές της. Αυτό αποτυπώνεται και στα ανανεωμένα γραφήματα, όπου παρουσιάζονται, κλιμακωμένα πλέον, τοπικά μέγιστα στην αρχή, μετά φθίνουν λόγω των αλλεπάλληλων διπλασιασμών και στο τέλος προκαλείται πάλι μέγιστο λόγω της επαναφοράς της συσχετιστικότητας σε μικρή τιμή και τον διπλασιασμό των καταχωρήσεων, ο οποίος επιφέρει μικρότερη κλιμάκωση.

\subsection{Προανάκληση}

Στο πείραμα της προανάκλησης δεν πραγματοποιήθηκαν μεταβολές στην χωρητικότητα της μνήμης ή στην συσχετιστικότητα, επομένως τα αποτελέσματα δεν χρειάζονται προσαρμογή.

\end{document}
